% Ejemplo de documento LaTeX
% Tipo de documento y tamaño de letra
\documentclass[12pt]{article}
% Preparando para documento en Español.
% Para documento en Inglés no hay que hacer esto.
\usepackage[spanish]{babel}
\selectlanguage{spanish}
\usepackage[utf8]{inputenc}
% EL titulo, autor y fecha del documento
\title{Tutorial breve de los comandos de Bash}
\author{Martin Alejandro Paredes Sosa}
% Aqui comienza el cuerpo del documento
\begin{document}
% Construye el título
\maketitle

\section{¿Qué es {\tt bash}?}

Bash es un interpretador de comandos utilizado sobre el sistema operativo Linux. 
Su función es de mediar entre el usuario y el sistema.

\section{Navegación}
Estos comandos son utilizados para moverse entre los directorios \\
\begin{tabular}{|c|l|l|}
\hline
Comando & Descripción & Ejemplo \\
\hline
pwd & Muestra el directorio en el que se esta trabajando & pwd:  /home/user\\ \hline
ls & Muestra un lista de contenido del directorio & ls: Dir1 Arc1\\ \hline
cd & Nos permite cambiar entre los directorios & cd [Destino] \\
\hline
\end{tabular} 

\section{Manipulacion de archivos}
Estos comando son los que te permiten manipular los archivos o directorios, como copiarlos, crearlos, borralos etre otras funciones.\\
\begin{tabular}{|c|c|l|}
\hline Comando & Descripción & Ejemplo \\ \hline
mkdir & Crea un nuevo directorio & mkdir [Nombre del directorio] \\ \hline
rmdir & Remueve un directorio exixtente & rmdir [Nombre del directorio] \\ \hline
touch & Crea un nuevo archivo & touch [Nombre del directorio] \\ \hline
cp & Copia un archivo & cp [origen][destino] \\ \hline
mv & Mueve un archivo & mv [origen] [Destino] \\ \hline
rm & Remueve un archivo & rm [Nombre del directorio] \\ \hline
cat & Te muestra el contenido de un archivo & cat [archivo] \\ \hline
vi & Te permite editar un archivo & vi [archivo] \\ \hline

\end{tabular} 
\subsection{Procesos}
\subsection{Temp}
-echo: Muestra un mensajes
user@bash: echo Alguien estuvo aqui

-less: Permite moverte en un archivo
user@bash: less MAPSFile

-chmod: Cambiar los permisos de un directorio o archivo
user@bash: chmod ug+rwx

-head: Ver las primera lineas de informcion
user@bash: head -5 MAPSFile.txt

-tail: Ver las ultimas lineas de informacion
user@bash: tail MAPSFile.txt

-sort: Organiza la informacion
user@bash: sort MAPSFile.txt

-nl: Muestra linea de numero antes de la informacion
user@bash: nl MAPSFile.txt

-wc: Muestra el numero de lineas, palabras y caracteres
user@bash: wc MAPSFile

-cut: Corta la informacion en campos y te meustra el campo espesificado
user@bash: cut MAPSFile.txt

-sed: Busca y remplaza en la informacion
user@bash:sed 's/Www/wow/g MAPSFile.txt

-uniq: Remueve lineas duplicadas
user@bash: uniq MAPSFile.txt

-tac: Muestra los datos en orden inverso
user@bash: tac MAPSFile.txt

-egrep: Se muestran lieneas de datos que coinciden con un patron
user@bash: egrep 'color'

-top: Ver los proceso que se estan ejecutando en tiempo real
user@bash: top

-ps: Consige un lista de los procesos que estan en ejecucion
user@bash: ps aux | grep 'wong'

-kill: Termina un proceso
user@bash: kill 7689

-jobs: Muestra una lista de t4rabajos que estan corriendo en un segundo plano
user@bash: jobs 

-fg: Muevo procesos de segundo plano a un primer plano
fg 4

-which: Te muestra el camino a un programa en particular
user@bash: which bash

\section{Manual}
Este comando es utilizado para conocer cual es su uso y además te muestra los parametros que el comando acepta. \\ 
\begin{tabular}{|c|l|l|}
\hline
Comando & Descripción & Ejemplo \\
\hline
man[comando] & Te muestra el uso del comando & man ls: mustra contenido\\ \hline
\end{tabular} 

% Nunca debe faltar esta última linea.
\end{document}
