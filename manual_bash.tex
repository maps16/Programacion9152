% Ejemplo de documento LaTeX
% Tipo de documento y tamaño de letra
\documentclass[12pt]{article}

% Preparando para documento en Español.
% Para documento en Inglés no hay que hacer esto.
\usepackage[spanish]{babel}
\selectlanguage{spanish}
\usepackage[utf8]{inputenc}

% EL titulo, autor y fecha del documento
\title{Tutorial breve de los comandos de Bash}
\author{Linus Torsvald}
\date{28 de Enero de 2015}

% Aqui comienza el cuerpo del documento
\begin{document}
% Construye el título
\maketitle

\section{¿Qué es {\tt bash}?}

Bash es un interpretador de comandos utilizado sobre el sistema operativo Linux.
Su función es de mediar entre el usuario y el sistema.

\section{Navegación}

Esta sección describe como se puede navegar entre archivos y directorios.

\section{Husmeando en el sistema}

Veremos 3 comandos:

\begin{itemize}
\item {\tt ls} (Lista los archivos y directorios) 
\item {\tt less} (Ver el contenido de archivos)
\item {\tt file} (Nos informa sobre el tipo de archivo)
\end{itemize}


\begin{enumerate}
\item {\tt ls} (Lista los archivos y directorios) 
\item {\tt less} (Ver el contenido de archivos)
\item {\tt file} (Nos informa sobre el tipo de archivo)
\end{enumerate}

\begin{tabular}{|c|l|l|}
\hline
Comando & Descripción & Ejemplo \\
\hline
ls & listar contenido de directorios & ls -al \\ \hline
less & Permite ver el contenido de archivos de texto & less aaa.txt \\ \\hline
file & Nos regresa el tipo de archivo & file aaa.txt \\
\hline
\end{tabular} 

\section{Y otra sección mas}
\subsection{Y una subsección}


% Nunca debe faltar esta última linea.
\end{document}
