% Ejemplo de documento LaTeX
% Tipo de documento y tamaño de letra
\documentclass[12pt]{article}
% Preparando para documento en Español.
% Para documento en Inglés no hay que hacer esto.
\usepackage[spanish]{babel}
\selectlanguage{spanish}
\usepackage[utf8]{inputenc}
% EL titulo, autor y fecha del documento
\title{Tutorial breve de los comandos de Bash}
\author{Martin Alejandro Paredes Sosa}
% Aqui comienza el cuerpo del documento
\begin{document}
% Construye el título
\maketitle

\section{¿Qué es {\tt bash}?}

Bash es un interpretador de comandos utilizado sobre el sistema operativo Linux.
Su función es de mediar entre el usuario y el sistema.

\section{Navegación}

\begin{tabular}{|c|l|l|}
\hline
Comando & Descripción & Ejemplo \\
\hline
pwd & Muestra el directorio en el que se esta trabajando & pwd:  /home/user\\ \hline
ls & Muestra un lista de contenido del directorio & ls: Dir1 Arc1\\ \hline
cd & Nos permite cambiar entre los directorios & cd [lugar] \\
\hline
\end{tabular} 


\section{Manual}

\begin{tabular}{|c|l|l|}
\hline
Comando & Descripción & Ejemplo \\
\hline
man[comando] & Te muestra el uso del comando & man ls: mustra contenido\\ \hline
\end{tabular} 

\section{Manipulacion de archivos}
\begin{tabular}{|c|c|l|}
\hline Comando & Descripción & Ejemplo \\ \hline
mkdir & Crea un nuevo directorio & mkdir [Nombre del directorio] \\ \hline
rmdir & Remueve un directorio exixtente & rmdir [Nombre del directorio] \\ \hline
touch & Crea un nuevo archivo & touch [Nombre del directorio] \\ \hline
cp & Copia un archivo & cp [origen][destino] \\ \hline
mv & Mueve un archivo & mkdir [Nombre del directorio] \\ \hline

\end{tabular}
\subsection{Y una subsección}


% Nunca debe faltar esta última linea.
\end{document}
